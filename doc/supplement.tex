\documentclass[]{article}
\usepackage[round]{natbib}

\usepackage{fullpage}
\usepackage{listings}
\usepackage{url}
\usepackage{authblk}
\usepackage{graphicx}
\usepackage{color}
\usepackage{booktabs}

% lorem ipsum dummy text
\usepackage{lipsum}

\lstset{language=Python}

% cross-reference with main text
\usepackage{xr}
\externaldocument{doc}

% local definitions
\newcommand{\comment}[1]{{\textcolor{red}{Comment: #1}}}
\newcommand{\aprcomment}[1]{{\textcolor{blue}{APR: #1}}}

\usepackage{xspace}
\newcommand{\GEVA}{\texttt{GEVA}\xspace}
\newcommand{\tsdate}{\texttt{tsdate}\xspace}
\newcommand{\relate}{\texttt{Relate}\xspace}


\begin{document}
\title{Supporting Information for ``An article template''}
\author[1,*]{The Author}
\affil[1]{Planet Earth}
\affil[*]{email@address.edu}
\date{\today}
\maketitle

\renewcommand{\thefigure}{S\arabic{figure}}
\renewcommand{\thetable}{S\arabic{table}}
\renewcommand{\theequation}{S\arabic{equation}}
\setcounter{figure}{0}
\setcounter{table}{0}
\setcounter{equation}{0}

\section{Supplemental methods}

\bibliographystyle{genetics}
\bibliography{doc}

\clearpage

\section{Tables and figures}

\begin{table}[h]
    \caption{
        \label{tab:recent-spectra}
        \textbf{Mutation profiles from the past 100 generations, compared to
        Iceland trios.} The most recent time bin for each method included the
        past $\approx150$ generations. When singletons were included (when
        using data from \tsdate and \relate), the spectra of estimated recent
        standing variation were unchanged. Note that \GEVA does not report ages
        for singletons.  While the three methods provide similar spectra from
        recent mutations, the spectrum from the Iceland pedigrees differs, in
        particular for the C$\rightarrow$A and C$\rightarrow$G classes. These
        differences are up to $2\%$ of the proportion among all mutations,
        which corresponds to an under- or over-count of up to $\sim20\%$ of
        C$\rightarrow$A and C$\rightarrow$G mutations, respectively.  This
        difference remains whether the spectrum is estimated from only
        mutations that were phased in \citet{jonsson2017parental} or from all
        mutations (phased and unphased).
    }
    \centering
    \begin{tabular}[t]{l|cccccc}
        \toprule
        Dataset & A$\rightarrow$C & A$\rightarrow$G & A$\rightarrow$T &
            C$\rightarrow$A & C$\rightarrow$G & C$\rightarrow$T \\
        \midrule
        \GEVA & 0.0946 & 0.3600 & 0.0886 & 0.1201 & 0.1057 & 0.2310 \\
        \tsdate & 0.0931 & 0.3579 & 0.0899 & 0.1146 & 0.1061 & 0.2384 \\
        \tsdate (w/singletons) & 0.0989 & 0.3598 & 0.0908 & 0.1168 & 0.1062 & 0.2275 \\
        \relate & 0.0991 & 0.3610 & 0.0863 & 0.1124 & 0.1038 & 0.2374 \\
        \relate (w/singletons) & 0.1002 & 0.3590 & 0.0921 & 0.1164 & 0.1060 & 0.2263 \\
        \midrule
        Trios (phased) & 0.0953 & 0.3649 & 0.0890 & 0.0960 & 0.1216 & 0.2332 \\
        Trios (all mutations) & 0.0962 & 0.3638 & 0.0923 & 0.0951 & 0.1202 & 0.2324 \\
        \bottomrule
    \end{tabular}
\end{table}


\begin{table}[h]
\caption{\label{tab:1kgpops}Some \citet{10002015global} populations.}
\centering
\begin{tabular}[t]{lll}
\toprule
Code & Description & Region\\
\midrule
ESN & Esan in Nigeria & Africa\\
GWD & Gambian in Western Divisions in the Gambia & Africa\\
LWK & Luhya in Webuye, Kenya & Africa\\
MSL & Mende in Sierra Leone & Africa\\
YRI & Yoruba in Ibadan, Nigeria & Africa\\
\addlinespace
CEU & Utah Residents (CEPH) with Northern and Western European Ancestry & Europe\\
GBR & British in England and Scotland & Europe\\
FIN & Finnish in Finland & Europe\\
IBS & Iberian Population in Spain & Europe\\
TSI & Toscani in Italia & Europe\\
\addlinespace
CDX & Chinese Dai in Xishuangbanna, China & East Asia\\
CHB & Han Chinese in Beijing, China & East Asia\\
CHS & Southern Han Chinese & East Asia\\
JPT & Japanese in Tokyo, Japan & East Asia\\
KHV & Kinh in Ho Chi Minh City, Vietnam & East Asia\\
\bottomrule
\end{tabular}
\end{table}

\clearpage

\begin{figure}[ht!]
    \centering
    \includegraphics[width=0.8\textwidth]{../plots/spectrum_history.geva.max_age.10000.pdf}
    \caption{
        \textbf{\GEVA-inferred mutation spectrum history.}
    }
    \label{fig:geva-spectra}
\end{figure}


\begin{figure}[ht!]
    \centering
    \includegraphics[width=0.8\textwidth]{../plots/spectrum_history.relate.max_age.10000.pdf}
    \caption{
        \textbf{\relate-inferred mutation spectrum history.}
    }
    \label{fig:relate-spectra}
\end{figure}


\begin{figure}[ht!]
    \centering
    \includegraphics[width=0.8\textwidth]{../plots/spectrum_history.tsdate.max_age.10000.pdf}
    \caption{
        \textbf{\tsdate-inferred mutation spectrum history.}
    }
    \label{fig:tsdate-spectra}
\end{figure}

\begin{figure}[ht!]
    \centering
    \includegraphics[width=0.8\textwidth]{../plots/spectrum_history.relate.max_age.10000.singletons.pdf}
    \caption{
        \textbf{\relate-inferred mutation spectrum history, including singletons.}
    }
    \label{fig:relate-spectra-singletons}
\end{figure}


\begin{figure}[ht!]
    \centering
    \includegraphics[width=0.8\textwidth]{../plots/spectrum_history.tsdate.max_age.10000.singletons.pdf}
    \caption{
        \textbf{\tsdate-inferred mutation spectrum history, including singletons.}
    }
    \label{fig:tsdate-spectra-singletons}
\end{figure}


\begin{figure}[ht!]
    \centering
    \includegraphics[width=0.8\textwidth]{../plots/inferred_generation_times.DM.geva.max_age.10000.pdf}
    \caption{
        \textbf{\GEVA-inferred generation time histories.}
    }
    \label{fig:geva-gen-times}
\end{figure}

\begin{figure}[ht!]
    \centering
    \includegraphics[width=0.8\textwidth]{../plots/inferred_generation_times.DM.relate.max_age.10000.pdf}
    \caption{
        \textbf{\relate-inferred generation time histories.}
    }
    \label{fig:relate-gen-times}
\end{figure}


\begin{figure}[ht!]
    \centering
    \includegraphics[width=0.8\textwidth]{../plots/inferred_generation_times.DM.tsdate.max_age.10000.pdf}
    \caption{
        \textbf{\tsdate-inferred generation time histories.}
    }
    \label{fig:tsdate-gen-times}
\end{figure}


\begin{figure}[ht!]
    \centering
    \includegraphics[width=0.8\textwidth]{../plots/goodness-of-fit.DM.geva.max_age.10000.pdf}
    \caption{
        \textbf{Prediction of mutation spectrum history from
        \GEVA-inferred generation times.}
    }
    \label{fig:geva-fit}
\end{figure}

\begin{figure}[ht!]
    \centering
    \includegraphics[width=0.8\textwidth]{../plots/goodness-of-fit.DM.relate.max_age.10000.pdf}
    \caption{
        \textbf{Prediction of mutation spectrum history from
        \relate-inferred generation times.}
    }
    \label{fig:relate-fit}
\end{figure}


\begin{figure}[ht!]
    \centering
    \includegraphics[width=0.8\textwidth]{../plots/goodness-of-fit.DM.tsdate.max_age.10000.pdf}
    \caption{
        \textbf{Prediction of mutation spectrum history from
        \tsdate-inferred generation times.}
    }
    \label{fig:tsdate-fit}
\end{figure}


\break

\end{document}
